\documentclass{article}
\usepackage[british]{babel}

\title{TEMA Test Engine for Developers}
\author{Antti Kervinen}

\begin{document}

\newcommand{\modulename}[1]{\texttt{#1}}
\newcommand{\classname}[1]{\textsc{\texttt{#1}}}
\newcommand{\methodname}[1]{\textbf{\texttt{#1}}}
\newcommand{\spec}[1]{\textbf{#1}}
\newenvironment{quot}{\begin{quote}\vspace{-3mm}}{\end{quote}}
\newcommand{\note}[1]{\flushleft\small\fbox{\begin{minipage}{0.8\linewidth}#1\end{minipage}}}


\section{Introduction}

This document specifies the most essential interfaces inside the TEMA
test engine.

\section{Model interface}

The model interface hides test modelling formalism. This allows using
several modelling formalisms with the test engine and even mixing
them.

\subsection{Classes and methods}

\classname{Model} class
\begin{itemize}
\item[] \methodname{loadFromFile}(filelikeobj)
  \begin{quot}
    This method is called before any other methods.\\
    \spec{Args:} filelikeobj is usually \classname{file} object. If
    not, it should at least implement the methods that are used for
    reading the ``file'' contents in the \methodname{loadFromFile}
    (typically
    \methodname{read} or \methodname{readline}).\\
    \spec{Returns:} -\\
    \spec{Exceptions:} may pass IOErrors and other problems in
    reading the file.
  \end{quot}
\item[] \methodname{getInitialState()}\\
  \begin{quot}
    \spec{Args:} -\\
    \spec{Returns:} a \classname{State} object that represents the starting state of the model.\\
    \spec{Exceptions:} -
  \end{quot}
\item[] \methodname{getActions()}\\
  \begin{quot}
    \spec{Args:} -\\
    \spec{Returns:} a list of \classname{Action} objects. The list
    contains all possible actions that may in the model.
    \\
    \spec{Exceptions:} -
  \end{quot}
\end{itemize}

\noindent{}\classname{State} class --- represents a state of the model
\begin{itemize}
\item[] \methodname{getOutTransitions}()
  \begin{quot}
    \spec{Args:} -\\
    \spec{Returns:} an iterable set of \classname{Transition} objects.
    The actions of the transitions specify the events that can occur
    in this state, and the destination states of the transitions
    specify the states after the events.
    \\
    \spec{Exceptions:} -
  \end{quot}
\item[] \methodname{equals}(\classname{State} object)
  \begin{quot}
    \spec{Args:} a \classname{State} object\\
    \spec{Returns:} a boolean value that is true iff the state given
    as a parameter represents the same state as this object.
    \\
    \spec{Exceptions:} -
  \end{quot}
\item[] \methodname{\_\_str\_\_}()
\end{itemize}

\noindent{}\classname{Transition} class --- represents a transition
from a state to another in the model
\begin{itemize}
\item[] \methodname{getAction}()
  \begin{quot}
    \spec{Args:} -\\
    \spec{Returns:} the \classname{Action} object associated with
    the transition.
    \\
    \spec{Exceptions:} -
  \end{quot}
\item[] \methodname{getSourceState}()
  \begin{quot}
    \spec{Args:} -\\
    \spec{Returns:} the \classname{State} object from which the
    transition can be executed. The state is the only state that
    returns the transition when its \methodname{getOutTransitions} is
    called.
    \\
    \spec{Exceptions:} -
  \end{quot}
\item[] \methodname{getDestState}()
  \begin{quot}
    \spec{Args:} -\\
    \spec{Returns:} the \classname{State} object to which the
    transition leads.
    \\
    \spec{Exceptions:} -
  \end{quot}
\item[] \methodname{equals}(\classname{Transition} object)
  \begin{quot}
    \spec{Args:} a \classname{Transition} object\\
    \spec{Returns:} a boolean value that is true iff the transition
    given as a parameter represents the same transition as this
    object.
    \\
    \spec{Exceptions:} -
  \end{quot}
\item[] \methodname{\_\_str\_\_}()
\end{itemize}

\noindent{}\classname{Action} class --- represents an action
\begin{itemize}
\item[] \methodname{toString}()
  \begin{quot}
    \spec{Args:} -\\
    \spec{Returns:} a string representation of the action.
    \\
    \spec{Exceptions:} -
  \end{quot}
\item[] \methodname{isKeyword}()
  \begin{quot}
    \spec{Args:} -\\
    \spec{Returns:} a boolean that is true iff the action is a
    keyword, that is, it should be sent to the SUT when the test is
    executed. Non-keyword actions can be exected without notifying the
    SUT.
    \\
    \spec{Exceptions:} -
  \end{quot}
\item[] \methodname{equals}(\classname{Action} object)
  \begin{quot}
    \spec{Args:} an \classname{Action} object\\
    \spec{Returns:} a boolean value that is true iff the action given
    as a parameter represents the same action as this object.
    \\
    \spec{Exceptions:} -
  \end{quot}
\item[] \methodname{\_\_str\_\_}()
\end{itemize}


\subsection{Example of usage}

LSTS model interface:
\begin{verbatim}
m = lstsmodel.Model()
m.loadFromFile( file("exampledata/BobTelephone.lsts") )

istate = m.getInitialState()
print "The initial state of BobTelephone.lsts is: %s" % (str(istate),)

for t in istate.getOutTransitions():
    print "Action %s can be executed in the initial state" % (t.getAction(),)
\end{verbatim}


\subsection{Available implementations}

\begin{itemize}
\item \modulename{lstsmodel.py} implements model interface for LSTS
  files.
\item \modulename{parallelmodel.py} implements model interface for
  parallel composition of any objects that implement the model
  interface.
\item \modulename{parallellstsmodel.py} implements model interface for
  parallel composition of LSTSes with \modulename{parallelmodel.py}.
\item \modulename{tracelogmodel.py} implements model interface for
  test logs. This can be used for repeating or investigating a
  previously run test: the test log can be used as a linear test model
  in the new test run or simulation / visualisation tool.
\item (not in real use at the time of writing this document)
  \modulename{prepostmodel.py} implements model interface for a
  precondition / postcondition formalism that is strongly based on
  Python.
\end{itemize}

\section{Coverage interface}

Coverage interface implements a function that maps sequences of
executed transitions to objects that tell how much of the required
coverage has been achieved.

\subsection{Classes and methods}

\noindent{}\classname{Coverage} class --- implements a test generation
algorithm
 
\begin{itemize}
\item[] \methodname{\_\_init\_\_}([covspec, [model=\classname{Model} object]])
  \begin{quot}
    \spec{Args:} Optional argument covspec, passes coverage language
    statement when requirement coverage in \modulename{clparser.py} is
    used. The other optional argument is the \classname{Model} object.\\
    \spec{Returns:} -\\
    \spec{Exceptions:} -
    \note{TODO: initialization of the coverage differs from other
      plugins. Testengine should be changed to accept arguments
      \texttt{--coverage=requirementcoverage --coverage-args="actions
        start\_aw.* then actions kw.*"} and \texttt{
        --coverage=statecoverage}. The arguments now given to the
      constructor should be given to the coverage with the standard
      setParameter() function calls.  }%note
  \end{quot}
\item[] \methodname{markExecuted}(\classname{Transition} object)
  \begin{quot}
    \spec{Args:} \classname{Transition} object that is considered executed.\\
    \spec{Returns:} -\\
    \spec{Exceptions:} -\\
  \end{quot}
\item[] \methodname{getPercentage}()
  \begin{quot}
    \spec{Args:} -\\
    \spec{Returns:} a coverage status object whose equality to 1.0
    must be testable and which should be comparable to other coverage
    status objects. This can be a floating point number, for example.\\
    \spec{Exceptions:} -\\
  \end{quot}
\item[] \methodname{pickDataValue}(set of possible data values)
  \begin{quot}
    If the coverage prefers using some data values over other, this
    method returns the value in the set. Otherwise it should return
    None. The returned value can be considered to be covered.\\
    \spec{Args:} a set of values that can be converted to string\\
    \spec{Returns:} None or a chosen value converted to string\\
    \spec{Exceptions:} -
  \end{quot}
\item[] \methodname{push}()
  \begin{quot}
    Stores the current coverage data to a stack. If the data is
    stored, \methodname{markExecuted}() and
    \methodname{pickDataValue}() can be called without affecting the
    stored data. Calling \methodname{pop}() restores the state of the
    previous \methodname{push}(). Push and pop calls can be
    nested. The purpose of these methods is to save memory, because
    test generation algorithms can avoid making of deep-copies of
    coverage objects. Coverage class implementations can use more
    sophisticated data structures to store the
    current state.\\
    \spec{Args:} -\\
    \spec{Returns:} -\\
    \spec{Exceptions:} -
  \end{quot}
\item[] \methodname{pop}()
  \begin{quot}
    Restores the coverage object to the stored state that is on the
    top of the stack and removes the state from the stack.\\
    \spec{Args:} -\\
    \spec{Returns:} -\\
    \spec{Exceptions:} IndexError (pop from emtpy list)
  \end{quot}
\end{itemize}

\subsection{Example of usage}

\begin{verbatim}
# init
c = coveragemodule.Coverage("actions .*", modelobject)
# see which action gives the best coverage
scores = []
current_state = modelobject.getInitialState()
for t in current_state.getOutTransitions():
    c.push()
    c.markExecuted(t)
    scores.append((c.getPercentage(), t.getAction()))
    c.pop()
scores.sort()
print "best score: %s with action %s" % (scores[-1],)
\end{verbatim}
Yet this \textbf{would} be better init:
\begin{verbatim}
c = coveragemodule.Coverage()
c.setParameter("model", modelobject)
c.setParameter("requirement", "actions start_aw.* then actions kw.*")
\end{verbatim}

\subsection{Command line usage}

Command line arguments related to coverage module
\begin{itemize}
\item \texttt{coverage}
    The coverage module. (If not given, the proper module is guessed based
    on the \texttt{coveragereq})
\item \texttt{coveragereq} Coverage language expression (REQUIRED)
\item \texttt{coveragereq-args} Optional arguments for coverage module
\end{itemize}

For example:
\begin{verbatim}
--coverage='clparser' \
--coveragereq='actions .*' \
--coveragereq-args=''
\end{verbatim}

\subsection{Available implementations}

\begin{itemize}
\item \modulename{clparser}
  Coverage language.
  Example of the language: \texttt{"actions .*"}
\item \modulename{findnewcoverage}
  Implements another (``findnew'') coverage language.
  Example of the language: \texttt{"findnew state"}
\item \modulename{altercoverage}
  Coverage module implementing ``find a new action word on another application''
  kind of heuristic.
  Example of the accepted language: \texttt{"app1 app2 app3 random"}.
  The last word in the language is either a number $N=2,3,...$ for
  $N$-coverage of applications or \texttt{random}.
\end{itemize}

\section{Guidance interface}

Test generation algorithms are written behind the guidance
interface. The algorithms are given a test model (that implements the
model interface) and a coverage object (that implements the coverage
interface) as input. In addition to them, several algorithm-specific
parameters can be passed to the object from the command line.

\subsection{Classes and methods}

\noindent{}\classname{Guidance} class --- implements a test generation
algorithm
\begin{itemize}
\item[] \methodname{setTestModel}(\classname{Model} object)
  \begin{quot}
    \spec{Args:}
    \spec{Returns:}
    \spec{Exceptions:}
  \end{quot}
\item[] \methodname{addRequirement}(\classname{Coverage} object)
  \begin{quot}
    \spec{Args:}
    \spec{Returns:}
    \spec{Exceptions:}
  \end{quot}
\item[] \methodname{prepareForRun}()
  \begin{quot}
    \spec{Args:}
    \spec{Returns:}
    \spec{Exceptions:}
  \end{quot}
\item[] \methodname{suggestAction}(\classname{State} object)
  \begin{quot}
    \spec{Args:} a \classname{State} object from the model interface.\\
    \spec{Returns:} \classname{Action} object (an action must always be returned)\\
    \spec{Exceptions:}
  \end{quot}
\item[] \methodname{markExecuted}(\classname{Transition} object)
  \begin{quot}
    \spec{Args:}
    \spec{Returns:}
    \spec{Exceptions:}
  \end{quot}
\end{itemize}


\subsection{Example of usage}

Command line arguments
\begin{verbatim}
--guidance-args='lookahead:15,randomseed:hop'
\end{verbatim}

cause the test engine core to call
\begin{enumerate}
\item \methodname{setParameter}(``lookahead'',``15'')
\item \methodname{setParameter}(``randomseed'',``hop'').
\end{enumerate}



\subsection{Available implementations}

\begin{itemize}
\item \modulename{randomguidance.py} implements a guidance that
  returns a random transition among the transitions leaving the given
  state.
\item \modulename{wrandomguidance.py} weighted random guidance. When
  transition objects in the model interface implement optional
  \methodname{getWeight}() method, this guidance algorithm can be used
  for selecting randomly an outgoing transition where the probability
  of becoming selected depends on the weight of the transition.
\item \modulename{gameguidance.py} calculates $n$ steps forward from
  the given state and returns the transition that leads to the best
  coverage after the $n$ steps.
\item \modulename{gameguidance-t.py} threading version of the
  previous: the possible steps are being continuously calculated in a
  separate thread in the background.
\item \modulename{greedyguidance.py} finds the shortest path improving
  coverage. Requires a \methodname{getExecutionHint} method from its
  coverage module.
\item \modulename{weightguidance.py} finds the best path up to a given search
  depth. Uses \methodname{transitionPoints} method of coverage object if it
  is available. Similar to \modulename{gameguidance} with a few extra
  parameters.
\item \modulename{tabuguidance.py} tries to execute actions/states/transitions
  that haven't been executed before.
\end{itemize}

\end{document}
