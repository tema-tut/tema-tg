\documentclass[a4paper,english]{article}
\usepackage[T1]{fontenc}
\usepackage[latin1]{inputenc}
\usepackage{babel}
\usepackage{graphics}

\title{\LARGE Socket Server Adapter Communication Protocol\\\vspace{.25cm}\large TEMA Unix package}

\author{Antti Kervinen}

\newcounter{hTeht}
\newcounter{cfree}
\newcounter{cypos}

%%% makrot sanomakaaviota varten
% tor (To Right) ja tol (To Left)
\newcommand{\tor}[1]{ % kehys apu->p�� {kehyksen teksti} {y-akseli}
  \put(0,\arabic{cypos}){\vector(4,0){140}}
  \setcounter{cfree}{\arabic{cypos}}
  \addtocounter{cfree}{2}
  \put(70,\arabic{cfree}){\rotatebox{0}{\makebox(0,0)[bc]{\scriptsize #1}}}
  \addtocounter{cypos}{-20} % jokainen askel tuo y:t� 30 pt alasp�in
}

\newcommand{\tol}[1]{ % kehys p��->apu {kehyksen teksti} {y-akseli}
  \put(140,\arabic{cypos}){\vector(-4,0){140}}
  \setcounter{cfree}{\arabic{cypos}}
  \addtocounter{cfree}{2}
  \put(70,\arabic{cfree}){\rotatebox{0}{\makebox(0,0)[bc]{\scriptsize #1}}}
  \addtocounter{cypos}{-20}
}

\newcommand{\comml}[1]{ % comment on the left hand side
  \setcounter{cfree}{\arabic{cypos}}
  \addtocounter{cfree}{20}
  \put(-115,\arabic{cfree}){\tbox{r}{#1}}
}

\newcommand{\commr}[1]{ % comment on the right hand side
  \setcounter{cfree}{\arabic{cypos}}
  \addtocounter{cfree}{20}
  \put(145,\arabic{cfree}){\tbox{l}{#1}}
}

\newcommand{\tbox}[2]{
  \parbox[t#1]{100 pt}{
    \scriptsize
    #2
  }
}


\begin{document}

\maketitle

\section{Introduction}

This document specifies the communication protocol implemented in TEMA
test engine's socket server adapter.

When the TEMA test engine wants to execute a keyword, it calls
\begin{quote}
  \texttt{sendInput(\textit{keyword string})}
\end{quote}
method of the adapter
chosen for the test run. The main task of the socket server adapter is
to forward these calls to remote clients over TCP/IP.

Inititally, the socket server adapter listens to a port for TCP/IP
connections. A client (the real adapter) connects to the port, says
``HELO'' and waits for acknowledgement. When received, the client goes
to a loop where it first requests a keyword, then executes it, and
finally reports the result to the adapter.

The client and the server talk by turns. That is, after sending a
message one must receive a message. Initially, client sends and the
server receives.

The protocol is designed to be used easily over a telnet connection.
Messages are in plain text, one message per line.

\section{Messages}

Message are plain text and are separated with linefeeds. Carriage
returns are also allowed: all whitespace should be stripped from the
end of the messages. Some messages may have parameters, which are
separated from the messages with one space character. The following
messages can be sent.

\begin{tabular}{lp{9.5cm}}
  & From client to server\\ \hline
  \texttt{HELO} & The first message from a client to a server after establishing a connection.\\
  \texttt{GET} & Keyword request\\
  \texttt{PUT} & Parameter required, either \texttt{TRUE} or \texttt{FALSE}. Returns the status of a keyword execution to the server.\\
  \texttt{BYE} & Client stops the test run. It should wait for \texttt{ACK}.\\
  \texttt{ERR} & Client reports unrecoverable error in the adapter layer.\\
  \texttt{ACK} & Positive acknowledgement: the last message (\texttt{BYE}) was understood.\\
  \texttt{LOG} & Parameter required: free text without line breaks. The parameter is printed to the test log on the server side. This message can be sent whenever it is client's turn to talk.
\end{tabular}

\begin{tabular}{lp{9.5cm}}
  & From server to client\\ \hline
  \texttt{ACK} & Positive acknowledgement from the server: the last message from the client was understood. \texttt{ACK} is followed by a parameter (a keyword) if the last message was \texttt{GET}.\\
  \texttt{NACK} & Negative response from the server: the last message was not understood.\\
  \texttt{BYE} & Server stops the test run. Client should send \texttt{ACK} before disconnecting.\\
  \texttt{ERR} & Server has detected an error (it found a difference between the expected and the observed behaviours). Client should repeat the last request.\\
\end{tabular}

\section{Client implementation}

A very simple client can be implemented as follows:

\begin{verbatim}
send HELO
receive ACK, otherwise quit with an error message

while true
    send GET
    receive a message
    if received BYE: send ACK and quit (test run ended)
    if received ERR: save information for debugging
    if received ACK:
        execute the keyword given as ACK parameter.
        repeat
            send PUT TRUE or PUT FALSE (status of the execution)
            receive a message
            if received BYE: send ACK and quit
            if received ERR: save information for debugging
        until received ACK
\end{verbatim}

Clients can use the test log on the server side. Whenever it is
client's turn to talk, it is allowed to send \texttt{LOG message}. The
socket server adapter immediately writes \texttt{message} to the
server log and sends \texttt{ACK}.

To interrupt a test run, a client sends \texttt{BYE}. To be sure
that the server understood, the client should wait for \texttt{ACK}.

\section{Communication example}

\begin{center}
  \begin{picture}(300,330)(-80,0)
    % viivat ymp�rille
    \put(0,310){\line(0,-1){340}}
    \put(140,310){\line(0,-1){340}}
    \put(0,325){\makebox(0,0)[c]{Client}}
    \put(140,325){\makebox(0,0)[c]{Socket server adapter}}
    
    % keskustelu
    \setcounter{cypos}{300}
    \tor{HELO} \comml{Client greets the server}
    \tol{ACK} \commr{Server sends ack}
    \tor{GET} \comml{Client requests the first keyword}
    \tol{ACK kw\_LaunchApp 'Anti-Virus'} \commr{Server sends ACK with
      a keyword}
    \tor{PUT TRUE} \comml{Client reports that keyword was successfully 
      executed}

    \tol{ACK} \commr{Server understood the message}
    \tor{GET} \comml{Client requests the second keyword}
    \tol{ACK kw\_VerifyText 'Anti-Virus'}
    \tor{PUT FALSE}
    \tol{ACK}
    \tor{GET} \comml{Client requests the third keyword}
    \tol{ERR}
    \commr{Server has detected an error: the last execution status should
      have been TRUE.}
    \tor{LOG debug data saved to file X}
    \comml{Client saves some data for debugging.
      The location of the data is stored to the server log}
    \tol{ACK}
    \tor{GET} \comml{Client re-requests the third keyword}
    \tol{BYE} \commr{Server decided not to continue the test run}
    \tor{ACK} \comml{Client acknowledges}
    
  \end{picture}
\end{center}


\end{document}
